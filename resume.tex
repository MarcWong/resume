% !TEX TS-program = xelatex
% !TEX encoding = UTF-8 Unicode
% !Mode:: "TeX:UTF-8"

\documentclass{resume}
\usepackage{zh_CN-Adobefonts_external} % Simplified Chinese Support using external fonts (./fonts/zh_CN-Adobe/)
%\usepackage{zh_CN-Adobefonts_internal} % Simplified Chinese Support using system fonts
\usepackage{linespacing_fix} % disable extra space before next section
\usepackage{cite}

\begin{document}
\pagenumbering{gobble} % suppress displaying page number
\name{\href{https://mrright.wang/}{王尧}}

\basicInfo{
  \email{yaowang95@pku.edu.cn} \textperiodcentered\ 
  \phone{(+86) 134-010-76331} \textperiodcentered\ 
  \faGithub \href{https://github.com/MarcWong/}{MarcWong} \textperiodcentered\ 
  \faGraduationCap \href{https://scholar.google.com/citations?user=X8je0QsAAAAJ}{Wang, Yao}
}
\section{\faGraduationCap\ 教育背景}
\datedsubsection{\textbf{北京大学}}{\textit{2017年9月 -- 2020年7月}}
\textit{理学硕士}\ 计算机软件与理论,总GPA \textbf{3.50 / 4.0}
\begin{itemize}[parsep=0.5ex]
    \item 核心专业课程:基于图像和视频的三维重建、计算视觉理论模型与方法、图形计算、深度学习技术与应用、人机交互理论与技术
    \item 荣获2018年度三好学生及斯伦贝谢奖学金
\end{itemize}
\datedsubsection{\textbf{北京大学}}{\textit{2013年9月 -- 2017年7月}}
\textit{理学学士}\ 智能科学与技术,\quad GPA \textbf{3.34 / 4.0} (全系排名 \textbf{8 / 35})
\begin{itemize}[parsep=0.5ex]
	\item 核心专业课程:模式识别导论、数字图像处理、算法设计与分析、计算机系统导论、数据结构与算法、集合论与图论、Web技术概论
	\item 荣获2015年度三好学生,2016年度三好学生标兵
\end{itemize}


\section{\faBookmark\ 发表论文}
\begin{itemize}
    \item \textbf{1$^{st}$-co Author} Semantic 3D Reconstruction with Learning MVS and 2D Segmentation of Aerial Images. Appl. Sci. 2020, 10, 1275. 
    \item \textbf{1$^{st}$-co Author} Large-scale structure from motion with semantic constraints of aerial images[C]//Chinese Conference on Pattern Recognition and Computer Vision (PRCV). Springer, Cham, 2018: 347-359.
	\item \textbf{2$^{nd}$ Author} Sobel Heuristic Kernel for Aerial Semantic Segmentation[C]//2018 25$^{th}$ IEEE International Conference on Image Processing (ICIP). IEEE, 2018: 3074-3078.
	\item \textbf{1$^{st}$ Author} Abstract TP55: Spatio-temporal Flow Tractography (SFT) for Evaluation of Collateral Patterns in Acute Stroke[J]. 2017.
	\item \textbf{4$^{th}$ Author} Tensor Voting Extraction of Vessel Centerlines from Cerebral Angiograms. In International Symposium on Visual Computing 2016 Dec 12 (pp. 35-44). Springer, Cham. 
\end{itemize}


\section{\faFlask\ 科研经历}

\datedsubsection{\textbf{北京大学}}{\textit{2016年9月 -- 2020年6月}}
\role{科研实习生}{主管: 陈毅松$ $副教授}
\begin{itemize}
    \item 理解三维立体视觉Structure from Motion和Multi-view Stereo的流程,通读并熟悉openMVS源码的深度图生成过程
    \item 理解二维语义分割任务的工作流,能够根据论文的网络结构图复现代码,掌握tensorflow和pytorch的使用
    \item 探索三维重建与语义分割任务的结合方式,实现三维点云K-近邻的语义稠密点云概率融合流程,并利用二维重投影语义图计算误差
	\item 主导标注无人机语义分割数据集UDD,利用图片二维语义对特征点匹配进行过滤,从而提升稀疏重建精度;相关工作投稿于\href{https://link.springer.com/content/pdf/10.1007\%2F978-3-030-03398-9_30.pdf}{\textbf{PRCV 2018}}
	\item 使用OpenMVS、R-MVSNet和Deeplab搭建语义重建框架,相关工作发表在\href{https://www.mdpi.com/2076-3417/10/4/1275/pdf}{\textbf{Applied Sciences}}期刊
\end{itemize}

\datedsubsection{\textbf{加州大学洛杉矶分校},  加利福尼亚, 美国}{\textit{2016年6月 - 9月}}
\role{科研实习生}{主管: Fabien Scalzo$ $副教授}
\begin{itemize}[parsep=0.5ex]
  \item 实现脑部核磁共振影像的数据处理及可视化,体素构建及血流量计算
  \item 熟练掌握MATLAB的图像处理函数(滤波器、形态学操作等)
\end{itemize}


\datedsubsection{\textbf{斯坦福大学}, 加利福尼亚, 美国}{\textit{2016年2月}}
\role{参赛选手}{}
\begin{itemize}[parsep=0.5ex]
  \item 从中国各大学的50+名候选人中脱颖而出,从分析用户需求入手,领导5人团队进行Photo Magic自动修图软件的开发,旨在强化修图软件的智能区域分割与滤镜选择
  \item 使用的技术栈包括PHP, python, openCV, Wolfram Alpha,主导团队完成面向赛会评委的产品展示工作
\end{itemize}


%%%%%%%%% 社会经历  %%%%%%%%%%%
\section{\faUsers\ 社会经历}

\datedsubsection{\textbf{北京大学信息科学技术学院学生会} }{\textit{2015年9月 - 2016年9月}}
\role{学生会主席}{}
\begin{itemize}[parsep=0.5ex]
  \item 创办北京大学黑客马拉松(HackPKU),吸引来自清华、NYU、哈工大等8所高校近200人参赛,获学院领导肯定,现已举办三届,成为信科学院的品牌活动
  \item 赞助企业有商汤、IBM、微软、绿盟、青云、友盟+、搜狐、计蒜客等近10家
  \item 活动被搜狐教育等媒体报导,作为访谈嘉宾登上了“从零道一”栏目
\end{itemize}


\datedsubsection{\textbf{北京三鼎科技有限公司(ofo)}, 北京 }{\textit{2014年6月 - 2015年4月}}
\role{总裁助理 | 前端工程师}{}
\begin{itemize}[parsep=0.5ex]
  \item ofo资深员工,经历种子、天使轮融资,对初创企业运作方式、团队管理有一定了解
  \item 使用Ruby on Rails框架(前端AngularJS)开发网站ofo.so,熟悉MVC的架构
\end{itemize}


\datedsubsection{\textbf{北京闪银奇异科技有限公司}, 北京 }{\textit{2017年4月 - 2017年9月}}
\role{前端工程师}{}
\begin{itemize}[parsep=0.5ex]
  \item 主持公众号”蜗壳私人空间”(后更名为盒子空间)的二期迭代工作,正式上线3个月内订单量破万,交付后服务平稳运行
\end{itemize}


\datedsubsection{\textbf{北京医典科技有限公司,青桦医学科技(北京)有限公司}, 北京 }{\textit{2017年7月 - 2018年5月}}
\role{CEO}

\begin{itemize}[parsep=0.5ex]
  \item 整合北医9899级校友资源,成立医管会,提升高净值人群的就医体验
  \item 研发基于NHS患者指南的全科医学问答微信公众号“病了吗”,致力于普及分级诊疗,解决在互联网上医学信息混乱、错误、难以查找的问题
\end{itemize}


\datedsubsection{\textbf{免单君科技有限公司}, 深圳}{\textit{2018年7月 -- 2019年10月}}
\role{CTO \& 前端工程师}

\begin{itemize}[parsep=0.5ex]
	\item 负责产品与技术团队的对接工作,管理5个人的前后端开发团队,掌握Gitlab、墨刀、Tower等代码维护、原型设计及团队效率工具的使用
    \item 了解Django Restful Framework,熟练使用Vue.js(mpvue、uniapp)进行小程序及网页开发,“免单君商城”、“免单君商户”两个微信小程序及对应后台平稳上线运营半年
\end{itemize}


\section{\faHeartO\ 获奖情况}

\datedsubsection{\textbf{三维重建挑战组二等奖},
中国虚拟现实与可视化产业技术创新战略联盟}{\textit{2019年11月}}
\begin{itemize}
	\item 第二届虚拟现实技术及应用创新大赛
\end{itemize}

\section{\faCogs\ 技能}
\begin{itemize}[parsep=0.5ex]
  \item 编程语言:  Python, MATLAB, C++, JavaScript, git, bash
  \item 语言: 中文, 英语 (托福98,CET-6), 德语 (A2)
  \item 其他技能及爱好: 唱歌、打击乐、足球
\end{itemize}

\end{document}
