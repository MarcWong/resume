% !TEX program = xelatex

\documentclass{resume}

\begin{document}
\pagenumbering{gobble} % suppress displaying page number

\name{Yao (Marc) Wang}

\basicInfo{
  \email{yaowang95@pku.edu.cn} \textperiodcentered\ 
  \phone{(+86) 134-010-76331} \textperiodcentered\ 
  \faGithub \href{https://github.com/MarcWong/}{MarcWong} \textperiodcentered\ 
%   \faGraduationCap \href{https://scholar.google.com/citations?user=X8je0QsAAAAJ}{Wang, Yao}
  \birthday{30.06.1995}  \textperiodcentered\ 
}

\section{\faGraduationCap\ Bildung}
\datedsubsection{\textbf{Peking Universität}, Beijing, China}{\textit{Sept. 2017 -- Jul. 2020}}
\textit{M.Sc.} Informatik, \quad GPA \textbf{3.50 / 4.0}

\begin{itemize}
	\item \textbf{Vertiefung:} \space
	Bild- und videobasierte 3D-Rekonstruktion, Computergestützte Vision, Erweiterte Grafikverarbeitung, Technik und Anwendung von Deep Learning, Mensch-Computer-Interaktion: Theorie und Techniken
	\item Ausgezeichneter Student mit Schlumberger Stipendium in 2018
\end{itemize}


\datedsubsection{\textbf{Peking Universität}, Beijing, China}{\textit{Sept. 2013 -- Jul. 2017}}
\textit{B.Sc.} Intelligente Wissenschaften und Technologie, \quad GPA \textbf{3.34 / 4.0} (Top \textbf{8 von 35})

\begin{itemize}
	\item \textbf{Vertiefung:} \space Einführung in die Mustererkennung, Digitale Bildverarbeitung, Algorithmusdesign und -analyse, Einführung in Computersysteme, Datenstrukturen und Algorithmen, Mengen- und Graphentheorie, Web-Software-Technologie
	\item Ausgezeichneter Student in 2015, 2016
\end{itemize}

\section{\faFlask\ Forschung}

\datedsubsection{\textbf{Peking Universität}, Beijing, China}{\textit{Sept. 2016 -- Jul. 2020}}
\role{Studentischer Forscher}{Supervisor: Prof. Yisong Chen}

\begin{itemize}
    \item Vertraute mit der Theorie der Kantenerkennung, Multi-View-Stereo und semantischer Segmentierung
    \item Kenntnisse auf dem Workflow semantischer Segmentierungs- und Tiefenschätzungsaufgaben, die in der Lage sind, eine pyTorch-Implementierung effizient bereitzustellen, abhängig von dem in diesem Dokument beschriebenen Netzwerkstrukturdiagramm
	\item Entwickelte das Protokoll und beschriftete den Datensatz \href{https://mrright.wang/UDD/}{\textbf{UDD}} für die semantische Segmentierung städtischer Drohnen
	\item Reduzierte die Feature-Matching-Fehler durch Segmentierungsinformationen, um die Leistung der Rekonstruktion mit spärlicher Punktwolke zu verbessern, die in \href{https://link.springer.com/content/pdf/10.1007\%2F978-3-030-03398-9_30.pdf}{\textbf{PRCV 2018}} veröffentlicht wurde
    \item Aufbau einer semantischen Rekonstruktionspipeline basierend auf OpenMVS, R-MVSNet und Deeplab, die in \href{https://www.mdpi.com/2076-3417/10/4/1275/pdf}{\textbf{Applied Sciences}} veröffentlicht wurde

\end{itemize}

\datedsubsection{\textbf{Universität von Kalifornien, Los Angeles} Kalifornien, die USA}{\textit{Jun. 2016 - Sept. 2016}}
\role{Praktikant}{Supervisor: Prof. Fabien Scalzo}

\begin{itemize}
  \item Vertraute mit der Voxelkonstruktion und Visualisierungsmethode für den Blutfluss in fMRI
  \item Vertraute mit dem MATLAB-Bildverarbeitungs-Toolbox, insbesondere Bildmorphing und Kantenfilterung
\end{itemize}


%%%%%%%%% Publikation  %%%%%%%%%%%
\section{\faBookmark\ Publikation}
\begin{itemize} \small
    \item \textbf{Co-Erstautor} Semantic 3D Reconstruction with Learning MVS and 2D Segmentation of Aerial Images. Appl. Sci. 2020, 10, 1275.
    \item \textbf{Co-Erstautor} Large-scale structure from motion with semantic constraints of aerial images[C]//Chinese Conference on Pattern Recognition and Computer Vision (PRCV). Springer, Cham, 2018: 347-359.
	\item \textbf{Zweiter Autor} Sobel Heuristic Kernel for Aerial Semantic Segmentation[C]//2018 25$^{th}$ IEEE International Conference on Image Processing (ICIP). IEEE, 2018: 3074-3078.
	\item \textbf{Erstautor} Abstract TP55: Spatio-temporal Flow Tractography (SFT) for Evaluation of Collateral Patterns in Acute Stroke[J]. 2017.
	\item \textbf{Vierter Autor} Tensor Voting Extraction of Vessel Centerlines from Cerebral Angiograms. InInternational Symposium on Visual Computing 2016 Dec 12 (pp. 35-44). Springer, Cham.
\end{itemize}


%%%%%%%%% Erfahrungen  %%%%%%%%%%%
\section{\faHeartO\ Erfahrung}

\datedsubsection{\textbf{Zweiter Preis} in der 3D Rekonstruktion Herausforderung Gruppe, Chinesische Strategische Allianz für Innovation in der virtuellen Realität und visuellen Technologie}{\textit{Nov. 2019}}
\begin{itemize}
	\item Der Zweite Virtual Reality Technologie und Anwendungsinnovation Wettbewerb
\end{itemize}

% \datedsubsection{\textbf{Excellent Grass-roots League Cadres}, Communist Youth League Beijing Committee}{\textit{2016}}
% \begin{itemize}
% 	\item Pioneer Cup Excellent Grass-roots League Cadres of Capital University and Secondary Vocational Colleges
% \end{itemize}

%%%%%%%%% Social Experience  %%%%%%%%%%%
% \datedsubsection{\textbf{Student Union of school of EECS}, Beijing, China}{\textit{Sept. 2015 -- Jun. 2016}}
% \role{Chairman}

% \begin{itemize}
% 	\item Founder of HackPKU, which attracted over 200 participants from PKU, THU, NYU and more universities.
% 	\item Applauded by the leaders of the college, it has been hosted for three times since and has become an iconic yearly event in our department.
%     \item The main sponsoring companies of the event include IBM, Microsoft, Sensetime, Nsfocus, QCloud, Umeng+ and Sohu.
% \end{itemize}

%%%%%%%%% Kenntnisse \& Fähigkeiten  %%%%%%%%%%%
\section{\faCogs\ Kenntnisse \& Fähigkeiten}

\begin{itemize}[parsep=0.5ex]
  \item Programmierungssprachen: Python, MATLAB, C++, JavaScript, git, bash
  \item Sprachen: Chinesisch (Muttersprache), Englisch (TOEFL 98), Deutsch (A2)
  \item Hobby: Singen, Schlagzeug, Fußball
\end{itemize}

\end{document}